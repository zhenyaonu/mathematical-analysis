\documentclass[12pt,leqno]{amsart}
\pagestyle{plain}
\usepackage{latexsym,amsmath,amssymb}
%\usepackage[notref,notcite]{showkeys}
\usepackage{amsfonts}
\usepackage{geometry}
\usepackage{graphicx}
\graphicspath{ {images/} }

\setlength{\oddsidemargin}{1pt}
\setlength{\evensidemargin}{1pt}
\setlength{\marginparwidth}{30pt} % these gain 53pt width
\setlength{\topmargin}{1pt}       % gains 26pt height
\setlength{\headheight}{1pt}      % gains 11pt height
\setlength{\headsep}{1pt}         % gains 24pt height
%\setlength{\footheight}{12 pt} 	  % cannot be changed as number must fit
\setlength{\footskip}{24pt}       % gains 6pt height
\setlength{\textheight}{650pt}    % 528 + 26 + 11 + 24 + 6 + 55 for luck
\setlength{\textwidth}{460pt}     % 360 + 53 + 47 for luck



\def\dsp{\def\baselinestretch{1.35}\large
\normalsize}
%%%%This makes a double spacing. Use this with 11pt style. If you
%%%%want to use this just insert \dsp after the \begin{document}
%%%%The correct baselinestretch for double spacing is 1.37. However
%%%%you can use different parameter.


\def\U{{\mathcal U}}

\begin{document}

\centerline{\bf Homework 5 for Math 1530}
\centerline{Zhen Yao}


\bigskip


\noindent
{\bf Problem 43.}
Prove that there is an increasing sequence of integers $a_1<a_2<a_3<\ldots$ such that for every $k\in\mathbb{N}$, the sequence
$\{\sin(ka_n)\}_{n=1}^\infty$ converges.
\begin{proof}
Every sequence in compact space has a convergent subsequence. And since $\sin n$ is dense $[-1,1]$, then $\sin n$ has a convergent subsequence as $\{\sin b_1, \sin b_2,\cdots \}$.Then, we can know that $\{\sin (2b_1), \sin (2b_2),\cdots\}$ has a convergent subsequence $\{\sin (2c_1), \sin (2c_2),\cdots\}$, where $2c_k\in\{2b_n\}$. So we have a sequence $\{c_n\}$ such that both $\{\sin c_n\}$ and $\{\sin(2c_n)\}$ converge as $n\to\infty$. Moreover, $\{\sin(3c_n)\}$ has a convergent subsequence $\{\sin(3d_1), \sin(3d_2),\cdots\}$. And we can continue this process and define $a_1=b_1, a_2=c_2,a_3=d_3,\cdots$, and we can know that $\{\sin(ka_n)\}, \forall k \in \mathbb{N}$ converges. 
\end{proof}

\medskip


\noindent
{\bf Problem 44.}
Fix $k\in \mathbb{N}$ and define
$$
z_n=\frac{1^k+2^k+\ldots+n^k}{n^{k+1}},
\quad
n=1,2,3,\ldots
$$
Prove that
$$
\lim_{n\to\infty} n\left(z_n-\frac{1}{k+1}\right)=\frac{1}{2}.
$$
\begin{proof}
We have $\lim_{n\to\infty} n\left(z_n-\frac{1}{k+1}\right) = \lim_{n\to\infty} \frac{(k+1)(1^k+\cdots+n^k)-n^{k+1}}{(k+1)n^k} = \lim_{n\to\infty} \frac{x_n}{y_n}$. Then we have 
\begin{align*}
    \lim_{n\to\infty} \frac{x_n-x_{n-1}}{y_n-y_{n-1}} & = \frac{(k+1)((1^k+\cdots+n^k)-n^{k+1}-(k+1)(1^k+\cdots+(n-1)^k)+(n-1)^{k+1}}{n^k(k+1)-(k+1)(n-1)^k} \\
    & = \lim_{n\to\infty} \frac{\frac{(k+1)k}{2}n^{k-1}+\binom{k}{2}n^{k-2}+\cdots}{(k+1)(kn^{k-1}-\binom{k}{2}n^{k-2})} \\
    & = \frac{1}{2}  
\end{align*}
According to Stolz theorem, $\lim_{n\to\infty}\frac{x_n}{y_n} = \lim_{n\to\infty} \frac{x_n-x_{n-1}}{y_n-y_{n-1}} = 1/2$. Thus, this series converges to $1/2$.
\end{proof}


\noindent
{\bf Problem 45.}
Let $\beta>0$ and $\{ u_n\}$ be a sequence of positive real numbers such that $\displaystyle\frac{u_{n+1}}{u_n}\leq\beta$ for every $n\in\mathbb{N}$. Prove that
$$
\limsup_{n\to\infty}\sqrt[n]{u_n}\leq \limsup_{n\to\infty}\left(\frac{u_{n+1}}{u_n}\right).
$$
\begin{proof}
For $\forall N > 0$, with $\frac{u_{n+1}}{u_n}\leq\beta$, we have $u_n \leq c_N \beta^{n-N} = c_N \beta^{-N} \beta^n$. Then we can have 
\begin{align*}
    &\sqrt[n]{u_n} \leq \sqrt[n]{c_N \beta^{-N}}  \beta \\
    \Rightarrow & \limsup_{n\to\infty} \sqrt[n]{u_n} \leq \beta
\end{align*}
Since $\frac{u_{n+1}}{u_n}\leq \beta$ for all $\beta$, then $\limsup_{n\to\infty} \sqrt[n]{u_n} \leq \limsup_{n\to\infty}\left(\frac{u_{n+1}}{u_n}\right)$
\end{proof}

\medskip

\noindent
{\bf Problem 46.}
Prove that if a sequence $(a_n)$ of real numbers is convergent to a finite limit, $\lim_{n\to\infty} a_n=g\in\mathbb{R}$, then
$$
\lim_{x\to\infty} e^{-x}\sum_{n=0}^\infty a_n\frac{x^n}{n!} = g.
$$
\begin{proof}
We have 
\begin{align*}
    \left|e^{-x} \sum^\infty_{n=0} a_n \frac{x^n}{n!} - g \right| & = \left|e^{-x} \sum^\infty_{n=0} (a_n-g) \frac{x^n}{n!}\right|
\end{align*}
Since $\lim_{n\to\infty}a_n=g$, then $\forall \varepsilon > 0$, there exists an $N>0$, such that $\forall n>N$, $|a_n-g|<\varepsilon$. Then, 
\begin{align*}
    \left|e^{-x} \sum^\infty_{n=0} a_n \frac{x^n}{n!} - g \right| & = \left|e^{-x} \sum^{N-1}_{n=0}(a_n-g) \frac{x^n}{n!}+e^{-x} \sum^{\infty}_{n=N}(a_n-g) \frac{x^n}{n!}\right| \\
    & \leq \left|e^{-x} \sum^{N-1}_{n=0}(a_n-g-\varepsilon) \frac{x^n}{n!} \right|+\left|e^{-x} \sum^{N-1}_{n=0}\varepsilon \frac{x^n}{n!} \right|+\left| e^{-x} \sum^{\infty}_{n=N}\varepsilon \frac{x^n}{n!}\right|\\
    & = \left|e^{-x} \sum^{N-1}_{n=0}(a_n-g-\varepsilon) \frac{x^n}{n!} \right|+\left| e^{-x} \sum^{\infty}_{n=0}\varepsilon \frac{x^n}{n!}\right| \\
    & = \left|e^{-x} \sum^{N-1}_{n=0}(a_n-g-\varepsilon) \frac{x^n}{n!} \right|+\varepsilon
\end{align*}
Also, for the first term on the right hand, we have
\begin{align*}
    \lim_{x\to\infty}\left|e^{-x} \sum^{N-1}_{n=0}(a_n-g-\varepsilon) \frac{x^n}{n!} \right| = \sum^{N-1}_{n=0} \lim_{x\to\infty}(a_n-g-\varepsilon) \frac{x^n}{e^x n!}= 0 
\end{align*}
Then we have $\left|e^{-x} \sum^\infty_{n=0} a_n \frac{x^n}{n!} - g \right| \leq \varepsilon$, which means the limit goes to $g$.
\end{proof}

\medskip




\noindent
{\bf Problem 47.}
Suppose that a sequence of functions $f_1,f_2,\ldots$ converges uniformly on $[0,1]$ to some function $f$. Suppose also that there is a constant $M$ such that
$|f_i(x)|\leq M$ for all $i$ and $x$. Prove that the sequence of squares $f_1^2,f_2^2,\ldots$ converges uniformly to $f^2$.
\begin{proof}
Since $f_1,f_2,\cdots$ converges uniformly on $[0,1]$, then for $\forall \varepsilon >0$, $\exists \delta > 0$, then $\forall x,y \in [0,1]$, if $|x-y|<\delta$, then $|f_i(x) - f_i(y)| < \frac{\varepsilon}{2M}$. Also, $f_1,f_2,\cdots$ converges to function $f$, then for the same $\varepsilon$ as before, there exists an $N>0$ such that $\forall n>N$, we have $|f_n(x)-f(x)|<\frac{\varepsilon}{2M}, \forall x\in[0,1]$. Then, if $|x-y|<\delta$ and $n>N$, we have
\begin{align*}
    & |f_i^2(x)-f_i^2(y)|\leq |f_i(x)-f_i(y)| |f_i(x)+f_i(y)| \leq 2M \frac{\varepsilon}{2M} = \varepsilon \\
    & |f_n^2(x)-f^2(x)| \leq |f_n(x)-f(x)| |f_n(x)+f(x)| \leq 2M \frac{\varepsilon}{2M} = \varepsilon
\end{align*}
Then we know that $f_1^2, f_2^2, \cdots$ converges uniformly to $f^2$.
\end{proof}

\medskip

\noindent
{\bf Problem 48.}
Prove that if $f:(0,1)\to\mathbb{R}$ is uniformly continuous, then there is a continuous function $F:[0,1]\to\mathbb{R}$ such that $F(x)=f(x)$ for all $x\in (0,1)$.
\begin{proof}
Consider the sequence $a_n = 1/n, n\in \mathbb{N}$ and $b_n = 1-1/n, n\in \mathbb{N}$, and we define 
\begin{align*}
    F(0) &= \lim_{n\to\infty}f(a_n) = A \\
    F(1) &= \lim_{n\to\infty}f(b_n) = B\\
    F(x) &= f(x), x\in(0,1)
\end{align*}
Now we prove that $F(x)$ defined above is continuous. It is obvious that $F$ is continuous on $(0,1)$. Now consider a sequence $\{x_n\}$ converges to $0$, and we need to prove that $\lim_{n\to\infty}F(x_n) = \lim_{n\to\infty}f(x_n) = A$. For $\forall \varepsilon > 0$, there exists an $N_1>0$, such that for $n>N_1$, we have $|f(a_n) - A|<\varepsilon/2$. Since $f(x)$ is uniformly continuous, then for the same $\varepsilon$, there exists $\delta>0$, such that $\forall x,y\in(0,1)$ if $|x-y|<\delta$, then $|f(x)-f(y)|<\varepsilon/2$. Since $\{x_n\}$ converges to $0$, and $a_n=1/n$, then there exists an $N_2$ such that $x_n<\delta$ and $a_n<\delta$, then we have $|x_n-a_n|<\delta$, and we have
\begin{align*}
    |f(x_n)-A|\leq |f(a_n) - A|+|f(x_n)-f(a_n)|<\frac{\varepsilon}{2}+\frac{\varepsilon}{2} = \varepsilon
\end{align*}
Then we proved that $F(x)$ is continuous at $x=0$, similarly, $F(x)$ is also continuous at $x=1$. Then $F$ is continuous.
\end{proof}

\medskip

\noindent
{\bf Problem 49.}
Prove that if $f:[0,\infty)\to\mathbb{R}$ is continuous and the limit $\lim_{x\to\infty} f(x)$ exists and is finite, then $f$ is uniformly continuous.
\begin{proof}
Suppose $f(x)$ is not uniformly continuous, then $\exists \varepsilon >0$, for $\forall \delta >0$ and $\exists x,y \in(0,\infty)$, such that if $|x-y|<\delta$, then $|f(x)-f(y)|\geq\varepsilon$. Set $\varepsilon$ be such that the above statement is true, and set $\delta = 1/n$. So we can find $x_n, y_n \in [0,\infty)$ such that $|x_n-y_n|<1/n$ and $|f(x_n)-f(y_n)|\geq \varepsilon$. For $x_n$, we can find a subsequence $\{x_{n_k}\}$ such that $\lim_{\k\to\infty} x_{n_k} = x_0$. And we have $|x_{n_k}-y_{n_k}|<1/n_k$, then we have $\lim_{\k\to\infty} y_{n_k} = x_0$. Hence, we have $\lim_{k\to\infty}|f(x_{n_k})-f(y_{n_k})|=|f(x_0)-f(x_0)|=0$, which is a contradiction. The proof is complete.
\end{proof}

\medskip

\noindent
{\bf Problem 50.}
Prove that if $f:\mathbb{R}\to\mathbb{R}$ is uniformly continuous, then there exist constants $a\geq 0$, $b\geq 0$ such that $\vert f(x)\vert \leq a\vert x\vert +b$ for all $x\in\mathbb{R}$.
\begin{proof}
Since $f:\mathbb{R}\to\mathbb{R}$ is uniformly continuous, then $\forall \varepsilon > 0$, there exists $\delta >0$, for $\forall x,y$, such that $|x-y|<\delta$, $|f(x)-f(y)|\leq \varepsilon$. This holds for $\varepsilon=1$. \\
\hspace*{3em}We can present $x$ as $x=m\delta +r$, where $0\leq r<\delta$, then we have 
\begin{align*}
    |f(x)-f(0)| = & |f(m\delta +r)-f(0)|\\
    = & |f(m\delta+r)-f((m-1)\delta +r)+f((m-1)\delta+r)-f((m-2)\delta+r)+\cdots \\
    & +f(2\delta+r)-f(\delta+r)+f(\delta+r)-f(r)+f(r)-f(0)| \\
    \leq & m + 1 \\
    \leq & \frac{1}{\delta}x-\frac{r}{\delta}+1 \\
    \leq & \frac{1}{\delta}|x|-\frac{r}{\delta}+1
\end{align*}
Then we have $|f(x)|\leq \frac{1}{\delta}|x|+f(0)+2$. Set $a=\frac{1}{\delta}$ and $b=f(0)+2$, and the proof is complete.
\end{proof}

\medskip

\noindent
{\bf Problem 51.}
Prove that there is no continuous function $f:\mathbb{R}\to\mathbb{R}$ such that $f(f(x)) = -x$ for all $x\in\mathbb{R}$.
\begin{proof}
We can know that $f$ is invertible, since $f(f(f(f(x)))) = f(f(-x)) = x$. Any invertible function is either increasing or decreasing. If $f$ is increasing, we take $x<y$, then we have $f(x)<f(y)$ and $f(f(x))=-x<f(f(y))=-y$, which is a contradiction. If $f$ is decreasing, we take $x>y$, then we have $f(x)>f(y)$ and $f(f(x))=-x<f(f(y))=-y$, which is a contradiction. Thus, $f$ cannot be a continuous function.
\end{proof}

\medskip

\noindent
{\bf Problem 52.}
Suppose that $f:(0,1)\to\mathbb{R}$ is continuous. Suppose also that there are two sequences $x_n,y_n\in (0,1)$ both convergent to $0$ such that $f(x_n)\to 0$, $f(y_n)\to 1$. Prove that there is a sequence $z_n\in (0,1)$ convergent to $0$ such that $f(z_n)\to 1/2$.
\begin{proof}
Since $f(x)$ is continuous and $\lim_{n\to\infty}f(x_n)=0, \lim_{n\to\infty}f(y_n)=y$, then there for $\forall \varepsilon >0$, there exists a $\delta>0$, such that $|x_n-0|<\delta, |y_n-0|<\delta$, then we have $|f(x_n)-0|<\varepsilon$ and $|f(y_n)-1|<\varepsilon$. Also $\{x_n\}, \{y_n\}\to 0$, then for the same $\delta >0$, there exists an $N>0$, such that $\forall n>N$, we have $|x_n-0|<\delta$ and $|y_n-1|<\delta$.\\
\hspace*{3em}Now we pick the $n=N+1$ and $\varepsilon<1/2$, and we have $f(x_n)<\delta = 1/2$ and $1/2<f(y_n)<3/2$. Since $f$ is continuous, there exists a $z$ between $x_n$ and $y_n$ such that $f(x_n)<f(z)=1/2<f(y_n)$. We denote this $z$ by $z_1$. Then we set $n=N+2$, then we can find $z_2$. Continue this process and we can get a sequence $\{z_n\}$ such that $\lim_{n\to\infty}z_n = 0$ and $\lim_{n\to\infty}f(z_n) = 1/2$.
\end{proof}

\medskip

\noindent
{\bf Problem 53.}
Let $f:[a,b]\to\mathbb{R}$ be continuous. Prove that the function
$$
g(x) = \sup_{t\in[a,x]}f(t)
$$
is continuous.
\begin{proof}
Since $f$ is continuous, then $\forall \varepsilon>0$, there exists a $\delta>0$, if $|x-y|<\delta$, then $|f(x)-f(y)|<\varepsilon$. Now we fix $y\in [a,b]$ and we assume $|x-y|<\delta$. \\
\hspace*{3em}Firstly, we consider $x\in(y,y+\delta)$. If $\sup_{t\in[a,x]}f(t) = f(x_0)$ for some $x_0\in(y,x)$, then we have two conditions. One is that $\sup_{t\in[a,y]}f(t) = f(y)$, then we have $|g(x)-g(y)|<\varepsilon$. The second one is that $\sup_{t\in[a,y]}f(t) = f(t_0)$ for some $t_0\in[a,y]$, and since $\sup_{t\in[a,x]}f(t) = f(x_0)$ for some $x_0\in(y,x)$, we have $f(y) < f(t_0)\leq f(x_0)$. Thus, we have $|g(x)-g(y)|=|f(x_0)-f(t_0)|<|f(x_0)-f(y)|<\varepsilon$. \\
\hspace*{3em}Similarly, we can know that $|g(x)-g(y)|<\varepsilon$ for $x\in(y-\delta, y)$. Then we proved that $g$ is continuous.
\end{proof}

\medskip

\noindent
{\bf Problem 54.}
A function $f:[0,1]\to \mathbb{R}$ is continuous and has the property that
$$
\lim_{x\to 0^+}\frac{f(x+1/3) + f(x+2/3)}{x} = 1.
$$
Prove that there is $x_0\in [0,1]$ such that $f(x_0) = 0$.
\begin{proof}
Denote $x=1/n, n\geq 3, n\in\mathbb{N}$, then we have 
\begin{align*}
    \lim_{n\to\infty} f\left(\frac{1}{n}+\frac{1}{3}\right) + f\left(\frac{1}{n}+\frac{2}{3}\right) = \frac{1}{n} = 0
\end{align*}
Take $n\to\infty$, and we have $f(1/3)+f(2/3)=0$. If both $f(1/3)$ and $f(2/3)$ are equal to zero, the the proof is complete. If not, then one of $f(1/3)$ and $f(2/3)$ is greater than zero, and the other is less than zero. Since $f$ is continuous, there must exist a $x_0$ such that $f(x_0)= 0$.
\end{proof}
\noindent
{\bf Method 2.}
\begin{proof}
We want to prove that $f(1/3)+f(2/3)=0$, and we only need to prove $f(1/3)f(2/3)<0$. Now we prove it by contradiction. \\
\hspace*{3em}Suppose $f(1/3)f(2/3)>0$, and we assume $f(1/3)>0$ and $f(2/3)>0$. Without losing generality, we assume $f(1/3)<f(2/3)$. Since $f$ is continuous, then for $\forall \varepsilon >0$, there exists a $\delta>0$, such that if $x\in(0,\delta)$, $|f(1/3)-f(x+1/3)|<\varepsilon$ and $|f(2/3)-f(x+2/3)|<\varepsilon$. Now we set $\varepsilon = \frac{f(1/3)}{2}$, then we have 
\begin{align*}
    f(x+1/3)>f(1/3)-\varepsilon=\frac{f(1/3)}{2}>0\\
    f(x+2/3)>f(2/3)-\varepsilon=\frac{f(1/3)}{2}>0
\end{align*}
which implies that, when $x<f(1/3)$, we have
\begin{align*}
    \frac{f(x+1/3)+f(x+2/3)}{x}>\frac{f(1/3)}{x} > 1
\end{align*}
And it is contradiction. Then $f(1/3)f(2/3)<0$.
\end{proof}

\medskip


\end{document}


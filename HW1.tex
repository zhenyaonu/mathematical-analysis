\documentclass[12pt,leqno]{amsart}
\pagestyle{plain}
\usepackage{latexsym,amsmath,amssymb}
%\usepackage[notref,notcite]{showkeys}
\usepackage{amsmath}
\usepackage{amsfonts}
\usepackage{geometry}
\usepackage{graphicx}
\graphicspath{ {images/} }
\usepackage{amssymb}

\usepackage{geometry}
\usepackage{graphicx}
\graphicspath{ {images/} }


\setlength{\oddsidemargin}{1pt}
\setlength{\evensidemargin}{1pt}
\setlength{\marginparwidth}{30pt} % these gain 53pt width
\setlength{\topmargin}{1pt}       % gains 26pt height
\setlength{\headheight}{1pt}      % gains 11pt height
\setlength{\headsep}{1pt}         % gains 24pt height
%\setlength{\footheight}{12 pt} 	  % cannot be changed as number must fit
\setlength{\footskip}{24pt}       % gains 6pt height
\setlength{\textheight}{650pt}    % 528 + 26 + 11 + 24 + 6 + 55 for luck
\setlength{\textwidth}{460pt}     % 360 + 53 + 47 for luck



\def\dsp{\def\baselinestretch{1.35}\large
\normalsize}
%%%%This makes a double spacing. Use this with 11pt style. If you
%%%%want to use this just insert \dsp after the \begin{document}
%%%%The correct baselinestretch for double spacing is 1.37. However
%%%%you can use different parameter.


\def\U{{\mathcal U}}

\begin{document}

\centerline{\bf Homework 1 for MATH 1530}
\centerline{Zhen Yao}

\bigskip

\medskip

\noindent
{\bf Problem 1.}
Use the equivalence
\begin{equation}
\label{Leq7}
p\wedge (q\vee r) \quad \equiv \quad (p\wedge q)\vee (p\wedge r)
\end{equation}
to prove
$$
p\vee (q\wedge r) \quad \equiv \quad (p\vee q)\wedge (p\vee r).
$$
To this end apply \eqref{Leq7} to $\neg p$, $\neg q$, $\neg r$ in
place of $p$, $q$, $r$, and negate the statement using De Morgan's Laws.
\begin{proof}
Place $\neg p$, $\neg q$, $\neg r$ into the first equation, and we can have
\begin{align*}
    (\neg p) \wedge \left( (\neg q) \vee (\neg r) \right) & \equiv (\neg p \wedge \neg q) \vee (\neg p \wedge \neg r)
\end{align*}
and we negate the statement by using De Morgan's Laws several times, which yields
\begin{align*}
    \neg \left( (\neg p) \wedge \left( (\neg q) \vee (\neg r) \right) \right) & \equiv \neg (\neg p \wedge \neg q) \wedge \neg (\neg p \wedge \neg r) \\
    \Rightarrow p \vee \neg \left( (\neg q) \vee (\neg r) \right) & \equiv (p \vee q) \wedge (p \vee r) \\
    \Rightarrow p \vee (q \wedge r) & \equiv (p \vee q) \wedge (p \vee r)
\end{align*}
\end{proof}


\medskip

\noindent
{\bf Problem 2.}
Negate the statement\footnote{This is a true statement known as uniform continuity of the function
$\sin x$. However, you are not asked to prove the statement only to negate it.}
$$
    \forall \varepsilon>0\ \ \exists \delta>0 \ \ \forall x\in\mathbb{R}\ \ \forall y\in\mathbb{R}
    \ \
    (|x-y|<\delta \ \Rightarrow \ |\sin x - \sin y|<\varepsilon).
$$
\begin{proof}
$$
    \exists \varepsilon>0\ \ \forall \delta>0 \ \ \exists x\in\mathbb{R}\ \ \exists y\in\mathbb{R}\ \
    (|x-y|<\delta \ \wedge \ |\sin x - \sin y| \geq \varepsilon).
    $$
\end{proof}


\medskip

\noindent
{\bf Problem 3.}
Negate the statement: {\em For all real numbers $x,y$ satisfying $x<y$, there
is a rational number $q$ such that $x<q<y$.} Formulate
the negation as a sentence and not as a formula involving quantifiers.
\begin{proof}
Formulate the statement into formula is that 
$$
    \forall x \in\mathbb{R}\ \ \forall y\in\mathbb{R}\ \exists q \in \mathbb{Q}\ \ ( x < y \Rightarrow x < q < y)
$$
The negation of this formula is 
$$
    \exists x \in\mathbb{R}\ \ \exists y\in\mathbb{R}\ \forall q \in \mathbb{Q}\ \ ( x < y \wedge ((q \leq x) \vee (q \geq y)) )
$$
So the sentence is: {\em There exist real numbers $x,y$ satisfying $x<y$, then for any rational number $q \in \mathbb{Q}$, $q$ satisfies the condition $q \leq x$ or $y \leq q$.}
\end{proof}


\medskip

\noindent
{\bf Problem 4.}
Use an argument by contradiction prove that $\sqrt{3}$ is irrational.
\begin{proof}
Let's assume that $\sqrt{3}$ is a rational number. Then $\sqrt{3} = p/q$ for some positive integers and we assume $p$ and $q$ have no common factors. Then we have 
\begin{align*}
    3 = \frac{p^2}{q^2}, \qquad \qquad p^2 = 3 q^2
\end{align*}
(1)Firstly, if $p$ is an even number, and the square of an even number is still even, so $3q^2$ is an even number. The $q^2$ is an even number since $3$ is odd, then we have $q$ is also an even number. Thus, $p$ and $q$ are all even number and have common factor $2$, which contradicts our assumption. \\
(2)Secondly, if $p$ is an odd number, and the square of an odd number is still odd, so $3q^2$ is an odd number. The $q^2$ is an odd number since $3$ is odd, then we have $q$ is also an odd number. Now both $p$ and $q$ are odd number, then we can set $p = 2n + 1$ and $q = 2m + 1$ where $n$ and $m$ are some positive integers. And we have
\begin{align*}
    4n^2 + 4n + 1 &= 3(4m^2 + 4m + 1) \\
    \Rightarrow 2n^2 + 2n &= 6m^2 + 6m + 1
\end{align*}
The left side is even and the right side is odd, which is impossible. Also this condition contradicts our assumption. So $\sqrt{3}$ is not a rational number.
\end{proof}



\medskip

\noindent
{\bf Problem 5.}
Prove the following statement\footnote{Compare with Example~1.12.}
$$
    \forall \varepsilon>0\ \ \exists n_0\in\mathbb{N}\ \ \forall n\in\mathbb{N} \quad
    (n\geq n_0\ \Rightarrow \ n^{-1}\leq \varepsilon).
$$
\begin{proof}
Suppose to the contrary that the statement is false, then its negation is true
\begin{align*}
    \exists \varepsilon>0\ \ \forall n_0\in\mathbb{N}\ \ \exists n\in\mathbb{N} \quad
    (n\geq n_0\ \wedge \ n^{-1} > \varepsilon)
\end{align*}
There exist a $\varepsilon > 0$ such that for every $n_0$, there is $n$ such that $n\geq n_0$ and $n^{-1} > \varepsilon$, and it is true for $n_0 = \frac{2}{\varepsilon}$. This means that for $n_0 = \frac{2}{\varepsilon}$, there is a $n$ such that
$$
    n \geq \frac{2}{\varepsilon} \qquad \textbf{and} \qquad \frac{1}{n} > \varepsilon \Rightarrow n < \frac{1}{\varepsilon}
$$
These two inequalities contradict each other. The proof is complete. 
\end{proof}


\medskip

\noindent
{\bf Problem 6.}
Find a mistake in the solution to Problem 9 provided on page 19 in my notes and write a correct solution.
\begin{proof}
(1)The mistake is that $\left|f(x)-f(y)\right| < \varepsilon$ for all $\varepsilon > 0$ does not imply that $\left|f(x)-f(y)\right| < 0$, we cannot have $\left|f(x)-f(y)\right| = 0$. So the proof is not correct.\\
\hspace*{2em}(2) The continuous functions satisfy the condition. clearly the continuous functions satisfy this conditions since it is defined as this way. Suppose that a function $f$ satisfies this condition, then we have for any $\varepsilon > 0$, and for any $x,y \in \mathbb{R}$, there exists a $\delta > 0$, such that 
\begin{align*}
    \left| x - y \right| < \delta \Rightarrow \left|f(x)-f(y)\right| < \varepsilon
\end{align*}
then we have 
\begin{align*}
    \lim_{x \rightarrow y} f(x) = f(y)
\end{align*}
since $y \in \mathbb{R}$ is arbitrary, so function $f$ is continuous in every point $y \in \mathbb{R}$. Thus, the function $f$ that satisfies the condition is a continuous function.
\end{proof}

\noindent
{\bf Problem 7.}
{ Prove that for any set $A$ and any family of sets $\{ A_i\}_{i\in I}$
$$
A\setminus \bigcup_{i\in I} A_i =\bigcap_{i\in I} (A\setminus A_i)\, ,
$$
$$
A\setminus \bigcap_{i\in I} A_i = \bigcup_{i\in I} (A\setminus A_i)\, .
$$}
\begin{proof}
(1)We have
\begin{align*}
    x \in A\setminus \bigcup_{i\in I} A_i & = x\in A \wedge \neg \left(x \in \bigcup_{i\in I} A_i\right) \\
    & = x\in A \wedge \neg \left(x \in A_1 \vee  \cdots \vee x \in A_i \cdots, i \in I \right) \\
    & = x\in A \wedge \left(x \notin A_1 \wedge \cdots \wedge x\notin A_i \cdots \right) \\
    & = (x \in A \wedge x \notin A_1) \wedge \cdots \wedge (x \in A \wedge x \notin A_i) \cdots \\
    & = (x \in A \setminus A_1) \wedge \cdots \wedge (x \in A \setminus A_i) \cdots \\
    & = x \in (A \setminus A_1) \wedge \cdots \wedge (A \setminus A_i) \\
    & = x \in \bigcap_{i\in I} (A\setminus A_i)
\end{align*}
\hspace*{2em}(2)We have
\begin{align*}
    x \in A\setminus \bigcap_{i\in I} A_i & = x\in A \wedge \neg \left(x \in \bigcap_{i\in I} A_i \right) \\
    & = x\in A \wedge \neg (x\in A_1 \wedge \cdots \wedge x\in A_i \cdots) \\
    & = x\in A \wedge (x\notin A_1 \vee \cdots \vee x\notin A_i \cdots) \\
    & = (x\in A \wedge x\notin A_1) \vee \cdots \vee(x\in A \wedge x\notin A_i) \cdots \\
    & = x \in (A \setminus A_1) \vee \cdots \vee x \in (A \setminus A_i) \cdots \\
    & = x \in \bigcup_{i\in I} (A\setminus A_i)
\end{align*}
The proof is complete.
\end{proof}

\medskip

\noindent
{\bf Problem 8.}
{ Prove that if $f:X\to Y$ is a function and $A_1,A_2,A_3,\ldots$ are subsets of $X$, then
$$
f\left(\bigcup_{i=1}^\infty A_i\right) =
\bigcup_{i=1}^\infty f(A_i)\, ,
$$
and
\begin{equation}
\label{Seq1}
f\left( \bigcap_{i=1}^\infty A_i\right) \subset
\bigcap_{i=1}^\infty f(A_i).
\end{equation}
Provide an example to show that we do not necessarily have equality in (\ref{Seq1})}
\begin{proof}
(1)First, if $y\in f\left(\bigcup_{i=1}^\infty A_i\right)$, then we can have $y = f(x)$ for some $x \in \bigcup_{i=1}^\infty A_i$. Then we can know there exist one subset $A_k$ such that $x \in A_k$, then we have $y = f(x) \in f(A_k) \subset \bigcup_{i=1}^\infty A_i$. We proved that
\begin{align*}
    & f\left(\bigcup_{i=1}^\infty A_i\right) \Rightarrow \bigcup_{i=1}^\infty f(A_i) \\
    & f\left(\bigcup_{i=1}^\infty A_i\right) \subset \bigcup_{i=1}^\infty f(A_i)
\end{align*}
If $y \in \bigcup_{i=1}^\infty f(A_i)$, then there exist one $A_m$ such that $y \in f(A_m)$. Then we have $y = f(x)$ for some $x \in A_m$. Since $A_m \subset \bigcup_{i=1}^\infty A_i$, then $y = f(x) \in f\left(\bigcup_{i=1}^\infty A_i\right)$. We proved that
\begin{align*}
    \bigcup_{i=1}^\infty f(A_i) \subset f\left(\bigcup_{i=1}^\infty A_i\right) 
\end{align*}
i.e.,
\begin{align*}
    f\left(\bigcup_{i=1}^\infty A_i\right) = \bigcup_{i=1}^\infty f(A_i)
\end{align*}
\hspace*{2em}(2)If $y \in f\left( \bigcap_{i=1}^\infty A_i\right)$, then $y = f(x)$ for some $x \in \bigcap_{i=1}^\infty A_i$. Thus, $x \in A_k$ for every $A_k, k = 1,2,3,\cdots$, so we have $y = f(x) \in f(A_k)$ for every $A_k$, and then $y = f(x) \in \bigcap f(A_i)$ which implies 
\begin{align*}
    & y \in f\left( \bigcap_{i=1}^\infty A_i\right) \Rightarrow \bigcap_{i=1}^\infty f(A_i) \\
    & y \in f\left( \bigcap_{i=1}^\infty A_i\right) \subset \bigcap_{i=1}^\infty f(A_i)
\end{align*}
Example: Let $f(x) = x^2, x\in \mathbb{R}$, and $A_1 = [-4,0], A_2 = [0,4]$. Then we have $f(A_1) = [0, 16]$ and $f(A_2) = [0, 16]$, so $f(A_1) \bigcap f(A_2)  = [0, 16]$. However, $A_1 \bigcap A_2 = \emptyset$, then $f(A_1 \bigcap A_2) = \emptyset$. Then we have, in this case, $f(A_1) \bigcap f(A_2) \nRightarrow f(A_1 \bigcap A_2)$.

\end{proof}

\medskip


\noindent
{\bf Problem 9.}
{ Prove that if $f:X\to Y$ is one-to-one and $A_1,A_2,A_3,\ldots$ are subsets of $X$, then
$$
f\left(\bigcap_{i=1}^\infty A_i\right) =
\bigcap_{i=1}^\infty f(A_i)\, .
$$}
\begin{proof}
If $y \in \bigcap_{i=1}^\infty f(A_i)$, then there exist one $x \in X$ such that $y = f(x) \in \bigcap_{i=1}^\infty f(A_i)$. Then we have $f(x) \in f(A_k)$ for every $k$, then we get $x \in \bigcap_{i=1}^\infty A_i$. Hence, $y = f(x) \in f\left(\bigcap_{i=1}^\infty A_i\right)$, which implies
\begin{align*}
    & \bigcap_{i=1}^\infty f(A_i) \Rightarrow f\left(\bigcap_{i=1}^\infty A_i\right) \\
    & \bigcap_{i=1}^\infty f(A_i) \subset f\left(\bigcap_{i=1}^\infty A_i\right)
\end{align*}
Since we already know that $f\left(\bigcap_{i=1}^\infty A_i\right) \subset \bigcap_{i=1}^\infty f(A_i)$, then we proved that, if $f$ is one-to-one function, then 
\begin{align*}
    \bigcap_{i=1}^\infty f(A_i) = f\left(\bigcap_{i=1}^\infty A_i\right)
\end{align*}
\end{proof}

\medskip

\noindent
{\bf Problem 10.}
{ Prove that $5^{2n}-1$ is divisible by $8$ for all $n\in \mathbb{N}$.}
\begin{proof}
(1)For $n = 1$, $5^2 - 1 = 24 = 3 \cdot 8$, which is divisible by $8$.\\
\hspace*{2em}(2)For $n > 1$, suppose that $5^{2n}-1$ is divisible by $8$. We need to prove that $5^{2(n+1)}-1$ is divisible by $8$. By assumption, that $5^{2n}-1 = 8k$ for some $k \in \mathbb{N}$. We have
\begin{align*}
    5^{2(n+1)}-1 & = 5^{2n}5^2 - 1 = 25 \cdot (8k + 1) - 1 = 25 \cdot 8k + 3 \cdot 8 \\
    & = 8 \cdot (25 k + 3)
\end{align*}
which is divisible by $8$.
\end{proof}


\noindent
{\bf Problem 11.}
{ Prove that
$\displaystyle
1 + \frac{1}{\sqrt{2}} + \frac{1}{\sqrt{3}} + \cdots + \frac{1}{\sqrt{n}}
\geq \sqrt{n}.
$}

\begin{proof}
(1)For $n = 1$, we have $1 \geq 1$, which is true.\\
\hspace*{2em}(2)Suppose the inequality is true for $n = k$, then we need to prove it for $k + 1$
\begin{align*}
    1 + \frac{1}{\sqrt{2}} + \cdots + \frac{1}{\sqrt{k}} + \frac{1}{\sqrt{k + 1}} \geq \sqrt{k} + \frac{1}{\sqrt{k + 1}} 
\end{align*}
Now all we need to prove is that $\sqrt{k} + \frac{1}{\sqrt{k + 1}} \geq \sqrt{k + 1}$, and we have
\begin{align*}
    \sqrt{k} + \frac{1}{\sqrt{k + 1}} - \sqrt{k + 1} & = \frac{1}{\sqrt{k + 1}} \left(\sqrt{k(k + 1)} + 1 - (k + 1)\right) \\
    & = \frac{1}{\sqrt{k + 1}} \left(\sqrt{k^2 + k} - k\right) \\
    & > 0
\end{align*}
It is easy to see that $\sqrt{k^2 + k} - k > 0$, then the proof is complete.
\end{proof}



\medskip

\noindent
{\bf Problem 12.}
{ Let $a_1,\ldots, a_n, b_1,\ldots, b_n$ be positive numbers.
Prove that
$$
\prod_{i=1}^n (a_i+b_i)^{1/n} \geq
\prod_{i=1}^n a_i^{1/n} +
\prod_{i=1}^n b_i^{1/n}\, .
$$
{\bf Hint:} Divide both sides by the expression on the left hand side and use the arithmetic-geometric mean inequality.}

\begin{proof}
Divide the right side of the equation by $\prod_{i=1}^n (a_i+b_i)$ and apply arithmetic-geometric mean inequality, then we have 
\begin{align*}
    & \prod_{i=1}^n \left(\frac{a_i}{a_i + b_i}\right)^{1/n} \leq \frac{1}{n} \sum^n_{i=1} \frac{a_i}{a_i + b_i} \\
    \text{and} \quad & \prod_{i=1}^n \left(\frac{b_i}{a_i + b_i}\right)^{1/n} \leq \frac{1}{n} \sum^n_{i=1} \frac{b_i}{a_i + b_i}
\end{align*}
Adding these two inequalities and we have 
\begin{align*}
    \prod_{i=1}^n \left(\frac{1}{a_i + b_i} \right)^\frac{1}{n} \left(\prod_{i=1}^n a_i^{1/n} + \prod_{i=1}^n b_i^{1/n} \right) \leq \frac{1}{n} \sum^n_{i=1} 1 = \frac{1}{n} n = 1
\end{align*}
And multiplying both sides with $\prod_{i=1}^n \left(a_i + b_i\right)^\frac{1}{n}$, we can have final result
\begin{align*}
    \prod_{i=1}^n a_i^{1/n} + \prod_{i=1}^n b_i^{1/n} \leq \prod_{i=1}^n \left(a_i + b_i\right)^\frac{1}{n}
\end{align*}
\end{proof}


\medskip

\noindent
{\bf Problem 13.}
Prove that Schwartz inequality
$$
\left|\sum_{i=1}^n a_i\, b_i\right| \leq \left(\sum_{i=1}^n a_i^2\right)^{1/2}\left(\sum_{i=1}^n b_i^2\right)^{1/2}.
$$

\begin{proof}
(1)For $n = 1$, we have $\|a_1 b_1\| \leq a_1 b_1$. It is true.\\
\hspace*{2em}(2)For $n = 2$, by arithmetic-geometric mean inequality, we have 
\begin{align*}
    (a_1 b_1 + a_2 b_2)^2 & = a_1^2 b_1^2 + 2a_1 a_2 b_1 b_2 + a_2^2 b_2^2\\
    & \leq a_1^2 b_1^2 + a_2^2 b_2^2 + (a_1^2 b_2^2 + a_2^2 b_1^2) \\
    & = (a_1^2 + a_2^2)(b_1^2 + b_2^2)
\end{align*}
Then the inequality holds for $n = 2$.\\
\hspace*{2em}(3)Suppose the inequality holds for $n = k > 2$, then we need to show that it still holds for $n = k + 1$. We have
\begin{align*}
    \left(\sum^{k+1}_{i=1} a_i b_i\right)^2 & = \left(\sum^{k}_{i=1} a_i b_i\right)^2 + 2 a_{k+1} b_{k+1} \left(\sum^{k}_{i=1} a_i b_i\right) + (a_{k+1}b_{k+1})^2 \\
    & \leq \left(\sum^{k}_{i=1} a_i^2 \right) \left(\sum^{k}_{i=1} b_i^2 \right) + 2 a_{k+1} b_{k+1} \left(\sum^{k}_{i=1} a_i b_i\right) + (a_{k+1}b_{k+1})^2
\end{align*}
Meanwhile, we also have
\begin{align*}
    \left(\sum^{k+1}_{i=1} a_i^2 \right) \left(\sum^{k+1}_{i=1} b_i^2 \right) & = \left(\sum^{k}_{i=1} a_i^2 \right) \left(\sum^{k}_{i=1} b_i^2 \right) + b_{k+1}^2 \left(\sum^{k}_{i=1} a_i^2 \right) + a_{k+1}^2 \left(\sum^{k}_{i=1} b_i^2 \right) + (a_{k+1}b_{k+1})^2
\end{align*}
Then we only have to prove that 
\begin{align*}
    b_{k+1}^2 \left(\sum^{k}_{i=1} a_i^2 \right) + a_{k+1}^2 \left(\sum^{k}_{i=1} b_i^2 \right) \geq 2 a_{k+1} b_{k+1} \left(\sum^{k}_{i=1} a_i b_i\right)
\end{align*}
By by arithmetic-geometric mean inequality, we have 
\begin{align*}
    \frac{b_{k+1}}{a_{k+1}} \left(\sum^{k}_{i=1} a_i^2 \right) + \frac{a_{k+1}}{b_{k+1}} \left(\sum^{k}_{i=1} b_i^2 \right) \geq 2 \sqrt{\left(\sum^{k}_{i=1} a_i^2 \right) \left(\sum^{k}_{i=1} b_i^2 \right)}
\end{align*}
multiplying both sides with $b_{k+1} a_{k+1}$ and we have 
\begin{align*}
    b_{k+1}^2 \left(\sum^{k}_{i=1} a_i^2 \right) + a_{k+1}^2 \left(\sum^{k}_{i=1} b_i^2 \right) & \geq 2 a_{k+1} b_{k+1} \sqrt{\left(\sum^{k}_{i=1} a_i^2 \right) \left(\sum^{k}_{i=1} b_i^2 \right)} \\
    & \geq 2 a_{k+1} b_{k+1} \left(\sum^{k}_{i=1} a_i b_i\right)
\end{align*}
where the the last step is from the assumption that the inequality holds for $n = k$. The proof is complete.
\end{proof}


\medskip

\noindent
{\bf Problem 14.}
Use the Schwarz inequality to prove that if $a_1,\ldots,a_n>0$, then
$$
\frac{n}{\frac{1}{a_1}+\ldots+\frac{1}{a_n}}\leq \frac{a_1+\ldots+a_n}{n}\, .
$$

\begin{proof}
By Schwarz inequality, we have 
\begin{align*}
    (a_1 + \cdots + a_n)\left(\frac{1}{a_1} + \cdots + \frac{1}{a_n} \right) & \geq \left(\sum^n_{i=1}a_i \frac{1}{a_i} \right)^2 \\
    & \geq n^2
\end{align*}
then we rearrange the inequality, which implies 
\begin{align*}
    \frac{a_1 + \cdots + a_n}{n} \geq \frac{n}{\frac{1}{a_1} + \cdots + \frac{1}{a_n}}
\end{align*}
The proof is complete.
\end{proof}







\end{document}

\medskip

\noindent
{\bf Problem 8.}
{ Prove that
$\displaystyle
1 + \frac{1}{\sqrt{2}} + \frac{1}{\sqrt{3}} + \cdots + \frac{1}{\sqrt{n}}
\geq \sqrt{n}.
$}

\medskip

\noindent
{\bf Problem 9.}
{ Let $a_1,\ldots, a_n, b_1,\ldots, b_n$ be positive numbers.
Prove that
$$
\prod_{i=1}^n (a_i+b_i)^{1/n} \geq
\prod_{i=1}^n a_i^{1/n} +
\prod_{i=1}^n b_i^{1/n}\, .
$$
{\bf Hint:} Use the arithmetic-geometric mean inequality.}




\end{document}
